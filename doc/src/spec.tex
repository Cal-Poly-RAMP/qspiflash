%%%%%%%%%%%%%%%%%%%%%%%%%%%%%%%%%%%%%%%%%%%%%%%%%%%%%%%%%%%%%%%%%%%%%%%%%%%%%%%%
%%
%% Filename: 	spec.tex
%%
%% Project:	Wishbone Controlled Quad SPI Flash Controller
%%
%% Purpose:	This LaTeX file contains all of the documentation/description
%%		currently provided with this Quad SPI Flash Controller.
%%		It's not nearly as interesting as the PDF file it creates,
%%		so I'd recommend reading that before diving into this file.
%%		You should be able to find the PDF file in the SVN distribution
%%		together with this PDF file and a copy of the GPL-3.0 license
%%		this file is distributed under.
%%		
%%
%% Creator:	Dan Gisselquist
%%		Gisselquist Technology, LLC
%%
%%%%%%%%%%%%%%%%%%%%%%%%%%%%%%%%%%%%%%%%%%%%%%%%%%%%%%%%%%%%%%%%%%%%%%%%%%%%%%%%
%%
%% Copyright (C) 2015,2018, Gisselquist Technology, LLC
%%
%% This file is part of the set of Wishbone controlled SPI flash controllers
%% project
%%
%% The Wishbone SPI flash controller project is free software (firmware):
%% you can redistribute it and/or modify it under the terms of the GNU Lesser
%% General Public License as published by the Free Software Foundation, either
%% version 3 of the License, or (at your option) any later version.
%%
%% The Wishbone SPI flash controller project is distributed in the hope
%% that it will be useful, but WITHOUT ANY WARRANTY; without even the implied
%% warranty of MERCHANTIBILITY or FITNESS FOR A PARTICULAR PURPOSE.  See the
%% GNU Lesser General Public License for more details.
%%
%% You should have received a copy of the GNU Lesser General Public License
%% along with this program.  (It's in the $(ROOT)/doc directory.  Run make
%% with no target there if the PDF file isn't present.)  If not, see
%% <http://www.gnu.org/licenses/> for a copy.
%%
%% License:	LGPL, v3, as defined and found on www.gnu.org,
%%		http://www.gnu.org/licenses/lgpl.html
%%
%%
%%%%%%%%%%%%%%%%%%%%%%%%%%%%%%%%%%%%%%%%%%%%%%%%%%%%%%%%%%%%%%%%%%%%%%%%%%%%%%%%
\documentclass{gqtekspec}
\project{Quad SPI Flash Controller}
\title{Specification}
\author{Dan Gisselquist, Ph.D.}
\email{zipcpu (at) gmail.com}
\revision{Rev.~1.0}
\begin{document}
\pagestyle{gqtekspecplain}
\titlepage
\begin{license}
Copyright (C) \theyear\today, Gisselquist Technology, LLC

This file is part of the set of Wishbone controlled SPI flash controllers
project.

The Wishbone SPI flash controller project is free software (firmware):
you can redistribute it and/or modify it under the terms of the GNU Lesser
General Public License as published by the Free Software Foundation, either
version 3 of the License, or (at your option) any later version.
%%
The Wishbone SPI flash controller project is distributed in the hope
that it will be useful, but WITHOUT ANY WARRANTY; without even the implied
warranty of MERCHANTIBILITY or FITNESS FOR A PARTICULAR PURPOSE.  See the
GNU Lesser General Public License for more details.
%%
You should have received a copy of the GNU Lesser General Public License
with this program.  If not, see \texttt{http://www.gnu.org/licenses/} for a
copy.
\end{license}
\begin{revisionhistory}
1.0 & 11/22/2018 & Gisselquist & Completely updated into a more universal set of controllers\\\hline
0.3 & 8/11/2016 & Gisselquist & Added information on the Extended Quad SPI controller\\\hline
0.2 & 5/26/2015 & Gisselquist & Minor spelling changes\\\hline
0.1 & 5/13/2015 & Gisselquist & First Draft \\\hline
\end{revisionhistory}
% Revision History
% Table of Contents, named Contents
\tableofcontents
\listoffigures
\listoftables
\begin{preface}
The genesis of this project was a desire to communicate with and program an
FPGA board without the need for any proprietary tools.  This includes Xilinx
JTAG cables, or other proprietary loading capabilities such as Digilent's
Adept program.  As a result, all interactions with the board need to take
place using open source tools, and the board must be able to reprogram itself.

After several iterations of this controller, it has finally been rebuilt into
a generic controller that will provide a Wishbone interface for any flash chip.
\end{preface}

\chapter{Introduction}
\pagenumbering{arabic}
\setcounter{page}{1}

This repository of serial flash controllers has now gone through several
iterations of development.

It was initially built to support a fairly generic Spansion flash part
with QSPI mode.  This initial interface supported four control registers,
for erasing the flash, configuring the flash, reading from the status register,
and reading the manufacturer's ID.  This original core worked wonderfully
well beyond the flash part it was initially targetted for, so the core got
wide usage.

A second core was rebuilt and added to the repository to support a Micron
flash, with the purpose of running at twice the clock rate (200MHz system
clock, 100MHz SPI clock).  Instead of four registers, this core supported
roughly 16 registers.  This core was only ever used on the Digilent
Arty board, since it had way too many features for general purpose use
and since it had a horrible latency.

A third core was built for the S6SoC project.  This third core was designed
for the absolute minimum amount of logic, and was very successful at it.
However, since this core was read-only, it has seen been eclipsed by the
current controllers.

When it came time to build yet another flash controller, I started getting
tired of building controllers that were specific to one flash or another.
This is the genesis of version one of  this project.

This project contains three basic flash controllers that are designed to be
completely general purpose.  There's a basic SPI controller, a Dual SPI
controller, and a Quad SPI controller.  All three controllers are designed
for high speed operation using an ODDR clock.  All three controllers may
optionally contain a configuration register to issue other-than-read commands.
Likewise, they may optionally allow pipelined accesses whereby multiple values
may be read without re-issuing either command or address.  Further, the
DSPI and QSPI controllers will also use an XIP mode by default.

Where the {\em universal} part of the flash controller designs breaks down
is in the initial startup.  A configuration option can be used to remove this
logic.  Alternatively, this startup logic is maintained within single
generic block, allowing for easy adjustment.

%%%%%%%%%%%%%%%%%%%%%%%%%
%%%%%%%%%%%%%%%%%%%%%%%%%
%%%%%%%%%%%%%%%%%%%%%%%%%
%%%%%%%%%%%%%%%%%%%%%%%%%
%%%%%%%%%%%%%%%%%%%%%%%%%
%%%%%%%%%%%%%%%%%%%%%%%%%
%%%%%%%%%%%%%%%%%%%%%%%%%

For a description of how the internals of each core work, feel free to browse
through the Architecture chapter.

The registers that control the cores are discussed in the Registers chapter.

As required, you can find a wishbone datasheet in Chapt.~\ref{chap:wishbone}.

The final pertinent information for implementing the cores is found in the
I/O Ports chapter, Chapt.~\ref{chap:ioports}.

As always, write me if you have any questions or problems.

\chapter{Architecture}\label{chap:arch}

\section{OPT\_PIPE}
\section{OPT\_CFG}
\section{OPT\_STARTUP}
\section{LGFLASHSZ}

\chapter{Operation}\label{chap:ops}

Each of these flash controller cores is designed to be as simple as possible
to implement.  This means that reading from the flash is by design easy,
but sending or receiving other commands is more involved.

\section{Reading}

Reading from the flash is designed to be as simple as reading from the
Wishbone bus.

\begin{itemize}
\item[SPI] Upon a read request, the SPI controller will issue a read request,
	{\tt 8'h03}, followed by the request address.  Once the data returns
	from the flash, the value is returned on the bus.

\item[DSPI] The DUAL spi controller is built around the understanding that
	some startup code has already issued a Dual I/O read command
	{\tt 8'hbb} with the mode byte set for XIP mode.  Ever after, a read
	request results in the address (not the command) sent to the flash,
	followed by the mode byte and then 8-dummy bits.  Values are then
	returned on completion.

\item[QSPI] The Quad SPI controller does the same as the Dual SPI controller,
	save that the command issued is the Quad I/O read command, {\tt 8'heb}.
\end{itemize}

\section{Pipelined Reading}
If the {\tt OPT\_PIPE} option parameter is set, then consecutive reads may be
made without sending the address or the read command multiple times.  Using
this mode, the various controllers can return a 32-bit value in 32 clocks (SPI
mode), 16 clocks (Dual SPI mode), or 8 clocks (Quad SPI mode).

\section{Startup}

Some flash chips and implementations have a special startup sequences that
need to be followed.  This is the purpose of the {\tt OPT\_STARTUP} parameter
and the startup section it enables.

The Dual and Quad SPI controllers have an additional startup need: they need
the initial Dual I/O read or Quad I/O read command to be issued with a
proper mode byte.  At the end of this command, the device will be placed into
an eXecute In Place (XIP) mode.  From the XIP mode, read requests may start
immediately with the address in Dual or Quad I/O mode for extra speed.

\section{Low-level Control}

Low level control can be accomplished through the use of the {\tt R\_FLASHCFG}
register.  This register will allow a software program to issue any command
to the flash and receive any result. Configuration commands are all 8-bit
commands.  In SPI mode, the command writes the eight given bits to the
{\tt MOSI} port, while reading the bits back from the {\tt MISO} port.
A write to this port will initiate the transaction.  A later read will return
the 8-bits read during that operation.

For the two high speed controllers, the {\tt R\_FLASHCFG} register can also
be used to place the controller into a high speed mode where either two
I/O lines or all four I/O lines are used together to send or read a message.
When a command is given for high speed operation, the command also includes
the direction of that operation.

While the {\tt R\_FLASHCFG} port is controlling the SPI controller, any
reads from the FPGA memory will silently fail.

Finally, when you are ready to return the controller to normal read
opeartions, the Dual and Quad SPI controllers need to be placed back into
Dual or Quad I/O Modes.

\section{Low Level Examples}

To use the low-level facilities, the higher speed flashes need to begin by
taking the flash controller out of normal operating read mode.  This can
be done by first taking control of the interface.

fpga->writeio(R\_FLASHCFG, F\_END);

This writes a {\tt 13'h1100} to the interface, placing it into configuration
control.

After this, writing 16-bits of zeros to the device in normal SPI mode will
take the device out of any higher speed read mode it may have been in.  This
equates to sending a dummy address (only every other or every fourth bit will
be read), followed by a one on the {\tt MOSI} channel when the mode byte would
have taken place.  Then, when the CS line is later deactivated,
the device will be out of any XIP mode and ready for commands.

fpga->writeio(R\_FLASHCFG, F\_RESET);
fpga->writeio(R\_FLASHCFG, F\_RESET);
fpga->writeio(R\_FLASHCFG, F\_END);

The device is now ready to accept any command the underlying SPI device
supports.

For example, if we wish to issue a READID command (8'h9f), we could simply
write

m_fpga->writeio(R\_FLASHCFG, CFG_USERMODE | 0x9f); // a 16'h109f

This will send the READID command to the device.  To read the command back,
we'll need to clock the device and read one byte at a time.  The following
therefore will read eight bits: one write request to clock the interface,
followed by a second read request to get the data returned from the interface.

m_fpga->writeio(R\_FLASHCFG, CFG_USERMODE | 0); // a 16'h1000
r = m_fpga->readio(R\_FLASHCFG);

The software flash driver will read and pack three more bytes before returning.

Finally, to return the device to normal operation, we'll neeed to issue a
highspeed read command.  The following shows a Quad I/O read command.

fpga->writeio(R_FLASHCFG, QUAD_IO_READ);

This is followed  by three (dummy) address bytes.

// 3 address bytes
fpga->writeio(R_FLASHCFG, CFG_USERMODE | CFG_QSPEED | CFG_WEDIR);
fpga->writeio(R_FLASHCFG, CFG_USERMODE | CFG_QSPEED | CFG_WEDIR);
fpga->writeio(R_FLASHCFG, CFG_USERMODE | CFG_QSPEED | CFG_WEDIR);

A mode byte, 8'ha0, is required to place the device into XIP mode upon
completion.

// Mode byte
fpga->writeio(R_FLASHCFG, CFG_USERMODE | CFG_QSPEED | CFG_WEDIR | 0xa0);

At this point, we need to change direction from write to read.
Most interfaces require a dummy byte before the read command actually begins.

// Read a dummy byte
fpga->writeio(R_FLASHCFG, CFG_USERMODE | CFG_QSPEED );

Were we to continue, we'd be reading from the flash.  Instead, let's end
our command.

// Close the interface
fpga->writeio(R_FLASHCFG, 0);

\section{External Software Control}

To make this easier, a universal flash driver software is provided with this
core.  This driver requires the use of a class implementing the DEVID
interface.  From this interface, the controller uses the {\tt readio}
and {\tt writeio} methods to access the controller.  Reading from the
controller is done using the {\tt readi} (read with increment) vector method.

This controller has methods {\tt take\_offline()}, to remove the Dual or
Quad SPI controllers from their high speed modes, as well as
{\tt place\_online()} to return them to their high speed modes upon completion.
The software uses the existence of either the {\tt DSPI\_FLASH} or the
{\tt QSPI\_FLASH} macro to know whether to place the flash in Dual or Quad
I/O modes upon completion.

The primary entry point for this controller is the {\tt write()} method.
The {\tt write()} method takes as an argument an address within the flash
memory space, a length, and a buffer.  It then tries to write that buffer
to that memory space.  To do this, it will first check to see if the flash
configuration is set properly by calling {\tt verify\_config()}.  If not, it
will try to {\tt set\_config()} and exit if a subsequent {\tt verify\_config()}
fails.  It then checks through the memory region given to determine if the
sector needs to be erased, or just programmed.

If any bits within the region need to switch from zero to one, the
{\tt write()} method will call the {\tt erase\_sector()} method with an
address of the sector that needs to be erased.  (It doesn't support sub-sector
erase calls.)  A {\tt verify} argument can also be given to double check that
the erase was successful if desired.

Following the (optional) {\tt erase\_sector()}, the {\tt write()} method will
call the {\tt page\_program()} method.  This method accepts an address, length,
and data pointer.  It will program the flash (turning ones to zeros), starting
at the given address, using length bytes starting at the data pointer.  As
with the {\tt erase\_sector()} command, a {\tt verify} flag may be passed to
check that the page program command was successful.

As the {\tt write()} method completes, it will issue a call to
{\tt place\_online()} to put the flash controller back into a mode where it
can read from the flash again.

\chapter{Registers}\label{chap:regs}

The three flash controller support two address spaces: a single configuration
and control register, and the flash memory itself.  Only one can be read
or written to at a time.  Writes to the flash memory will quietly fail with
no error.  Reads from the flash memory while the configuration port are active
will also quietly fail with no error.  Reads from the configuration port
will return undefined values unless the port is in configuration mode.

Table.~\ref{tbl:reglist} shows this one register.
\begin{table}[htbp]
\begin{center}
\begin{reglist}
FLASHCFG & 0 & 13 & R/W & Configuration register, for low-level control \\\hline
\end{reglist}
\caption{One register for all three controllers}\label{tbl:reglist}
\end{center}\end{table}

The bits within this configuration register are shown in Table~\ref{tbl:bits}.
\begin{table}[htbp]
\begin{center}
\begin{bitlist}
13-31 & & Reserved.  These bits are ignored on read and write
	\\\hline
12 & R/W & User mode.  True if the port being controlled by the FLASHCFG
	register.
	\\\hline
11 & R/W & QUAD speed.  True if the controller supports QUAD I/O mode, and if
	the port has been set to high speed QUAD mode.
	\\\hline
10 & R/W & Dual speed.  True if the controller supports DUAL I/O mode, and if
	the port has been set to high speed DUAL mode.
	\\\hline
9 & R/W & High speed direction.  True if the pins of the bus are all set for
	writing, false if they are all high impedence for reading.  This
	pin is read-only for the SPI flash controller.
	\\\hline
8 & R/W & CS\_n.  If zero, the CS pin is active (low).
	\\\hline
0--7 & R/W & These are data lines.  A write to the SPI interface with bit 8 low
	will write these bits to the interface using SPI mode.  Likewise,
	for the DUAL and QUAD interfaces, a write to the register with bit
	8 low, bit 12 high, and the speed and direction bits set will write
	these bits at high speed across the port.  If the speed bit isn't set,
	the direction bit will be ignored and the bits will be written using
	SPI mode.  Once the operation is completed, any resulting bits from
	SPI mode, or from a read operation where the direction bit was clear,
	may be read from these bits of this register.
	\\\hline
\end{bitlist}
\caption{FLASHCFG bit definitions}\label{tbl:bits}
\end{center}\end{table}

Taking the device off line in order to issue a command is slightly different
between the SPI controller and the higher speed controllers.  With the
SPI controller, all that needs to be done is to write to the configuration
register with bit eight low.  This will activate the chip-select and clock
the eight data bits out of the port.  A subsequent read may read the bits
read during this time.

With the higher speed controllers, you also need to set the user override
mode bit.  Hence, if bit 12 is high and bit 8 is low, a message will be
sent or received from the device.  If the speed flag is left low, the
message will be sent in normal SPI mode.
	
If bit 12 is high and bit 8 is high, a high speed device will be removed
from normal operation, but nothing will be sent to it.  Setting thedual or
quad speed flag at thte same time will allow you to read back which of the
two implementations was built into the design.  If the Dual Speed bit is set,
then Dual controller is implemented within the design.  If the Quad speed bit
is set, then the quad controller is implemented within the design.  Sadly,
this method will not work with the SPI controller simply because commands with
bit 8 high will be ignored, and commands with bit 8 low will send a message
across the SPI channel.  For this reason, the software controller
uses the {\tt DUAL_SPI} and {\tt QUAD_SPI} macros to determine speed
instead.



\chapter{Wishbone Datasheet}\label{chap:wishbone}
Tbl.~\ref{tbl:wishbone} is required by the wishbone specification, and so 
it is included here.
\begin{table}[htbp]
\begin{center}
\begin{wishboneds}
Revision level of wishbone & WB B4 \\\hline
Type of interface & Slave, (Pipelind) Read/Write \\\hline
Port size & 32--bit \\\hline
Port granularity & 32--bit \\\hline
Maximum Operand Size & 32--bit \\\hline
Data transfer ordering & Big Endian \\\hline
Clock constraints & Most serial flash chips require 100MHz or
	slower\\\cline{2-2}
Signal Names & \begin{tabular}{ll}
		Signal Name & Wishbone Equivalent \\\hline
		{\tt i\_clk}   & {\tt CLK\_I} \\
		{\tt i\_reset} & {\tt CLK\_I} \\
		{\tt i\_wb\_cyc} & {\tt CYC\_I} \\
		{\tt i\_wb\_stb} & {\tt STB\_I} \\
		{\tt i\_cfg\_stb} & {\tt STB\_I} \\
		{\tt i\_wb\_we} & {\tt WE\_I} \\
		{\tt i\_wb\_addr} & {\tt ADR\_I} \\
		{\tt i\_wb\_data} & {\tt DAT\_I} \\
		{\tt o\_wb\_ack} & {\tt ACK\_O} \\
		{\tt o\_wb\_stall} & {\tt STALL\_O} \\
		{\tt o\_wb\_data} & {\tt DAT\_O}
		\end{tabular}\\\hline
\end{wishboneds}
\caption{Wishbone Datasheet for the basic Flash controller}\label{tbl:wishbone}
\end{center}\end{table}

\chapter{Clocks}\label{chap:clocks}

These cores are based upon the existence of an ODDR controller for the clock.
Using this ODDR part, it is possible to run the controller SCK at the system
clock rate.  Most flash parts, however, are limited to roughly 108MHz or so.
A safe clock speed limit should therefore be about 100MHz.
See Table.~\ref{tbl:qspi-clocks}.
\begin{table}[htbp]\begin{center}\begin{clocklist}
i\_clk & External & 100 & & System clock.\\\hline
\end{clocklist}
\caption{Clock speeds}\label{tbl:qspi-clocks}
\end{center}\end{table}

\chapter{I/O Ports}\label{chap:ioports}
There are two primary interfaces that these cores support: a wishbone
interface, and the interface to the top level I/O's.  Both of these have
their own section in the I/O port list.  For the purpose of this table,
the wishbone interface is listed in Tbl.~\ref{tbl:iowishbone}, and the
flash interface is listed in Tbl.~\ref{tbl:ioqspi}.  The two lines that
don't really fit this classification are found in Tbl.~\ref{tbl:ioother}.
\begin{table}[htbp]
\begin{center}
\begin{portlist}
i\_wb\_cyc & 1 & Input & Wishbone bus cycle wire.\\\hline
i\_wb\_data\_stb & 1 & Input & Wishbone strobe, when the access is to the data 
		memory.\\\hline
i\_wb\_cfg\_stb & 1 & Input & Wishbone strobe, for when the access is to 
	the configuration register\\\hline
i\_wb\_we & 1 & Input & Wishbone write enable, indicating a write interaction
		to the bus.  Writes tot he memory interface will be quietly
		ignored\\\hline
i\_wb\_addr & 21 & Input & Wishbone address.  When accessing the configuration,
		register, these address bits are ignored.\\\hline
i\_wb\_data & 32 & Input & Wishbone bus data register.  Used by the
		configuration register only.\\\hline
o\_wb\_ack & 1 & Output & Return value acknowledging a wishbone write, or
		signifying valid data in the case of a wishbone read request.
		\\\hline
o\_wb\_stall & 1 & Output & Indicates the device is not yet ready for another
		wishbone access, effectively stalling the bus.\\\hline
o\_wb\_data & 32 & Output & Wishbone data bus, returning data values read
		from the interface.\\\hline
\end{portlist}
\caption{Wishbone I/O Ports}\label{tbl:iowishbone}
\end{center}\end{table}

While this core is wishbone compatible, there was one necessary change to
the wishbone interface to make this possible.  That was the split of the
strobe line into two separate lines.  The first strobe line, the data strobe,
is used when the access is to data memory--such as a read or write (program)
access.  The second strobe line, the control strobe, is for reads and writes
to one of the four control registers.  By splitting these strobe lines,
the wishbone interconnect designer may place the control registers in a
separate location of wishbone address space from the flash memory.  It is
an error for both strobe lines to be on at the same time.

With respect to the higher speed interfaces, one piece of glue logic
is necessary to tie the flash I/O to the in/out port at the top
level of the device.  Specifically, the following two lines must be added at
the top level.  First, for dual I/O,
\begin{tabbing}
assign {\tt io\_dspi\_dat} = \= (\~{\tt qspi\_mod[1]})?(\{1'bz,{\tt dspi\_dat[0]}\}) \hbox{\em // Serial mode} \\
	\> :(({\tt dspi\_bmod[0]})?(2'bzz):({\tt dspi\_dat[1:0]}));
		\hbox{\em // Dual mode}
\end{tabbing}
and then for quad I/O,
\begin{tabbing}
assign {\tt io\_qspi\_dat} = \= (\~{\tt qspi\_mod[1]})?(\{2'b11,1'bz,{\tt qspi\_dat[0]}\}) \hbox{\em // Serial mode} \\
	\> :(({\tt qspi\_bmod[0]})?(4'bzzzz):({\tt qspi\_dat[3:0]}));
		\hbox{\em // Quad mode}
\end{tabbing}
These provide the transition between the input and output ports used by this 
core, and the bi--directional inout ports used by the actual part.  Further,
because the two additional lines are defined to be ones during serial I/O
mode, the hold and write protect lines are effectively eliminated in this
design in favor of faster speed I/O (i.e., Quad I/O).

Finally, the clock line is not specific to the wishbone bus, and the interrupt
line is not specific to any of the above.  These have been separated out here.
\begin{table}[htbp]
\begin{center}
\begin{portlist}
i\_clk & 1 & Input & The 100~MHz clock driving all SPI interactions.\\\hline
\end{portlist}
\caption{Other I/O Ports}\label{tbl:ioother}
\end{center}\end{table}
% Appendices
% Index
\end{document}


